%%%%%%%%%%%%%%%%%%%%%%%%%%%%%%%%%%%%%%%%%
% Medium Length Graduate Curriculum Vitae
% LaTeX Template
% Version 1.1 (9/12/12)
%
% This template has been downloaded from:
% http://www.LaTeXTemplates.com
%
% Original author:
% Rensselaer Polytechnic Institute (http://www.rpi.edu/dept/arc/training/latex/resumes/)
%
% Important note:
% This template requires the res.cls file to be in the same directory as the
% .tex file. The res.cls file provides the resume style used for structuring the
% document.
%
%%%%%%%%%%%%%%%%%%%%%%%%%%%%%%%%%%%%%%%%%

%----------------------------------------------------------------------------------------
%	PACKAGES AND OTHER DOCUMENT CONFIGURATIONS
%----------------------------------------------------------------------------------------

\documentclass[margin, 10pt]{res} % Use the res.cls style, the font size can be changed to 11pt or 12pt here

\usepackage{helvet} % Default font is the helvetica postscript font
%\usepackage{newcent} % To change the default font to the new century schoolbook postscript font uncomment this line and comment the one above

\usepackage{wrapfig}
\usepackage{graphicx}
\usepackage{caption}
\usepackage{subcaption}
\usepackage[danish]{babel}
\usepackage{xcolor}
\usepackage{blindtext}
\usepackage{hyperref}
\usepackage[utf8]{inputenc}

\setlength{\textwidth}{5.1in} % Text width of the document

\begin{document}

%----------------------------------------------------------------------------------------
%	NAME AND ADDRESS SECTION
%----------------------------------------------------------------------------------------
\fboxsep=2pt%padding thickness
\fboxrule=1pt%border thickness

\begin{figure}[htp]
\begin{subfigure}[b]{0.865\resumewidth}
\moveleft\hoffset\vbox{\huge\bf Jens Hegner Stærmose} % Your name at the top

\moveleft\hoffset\vbox{\hrule width 0.867\resumewidth height 1pt}\smallskip % Horizontal line after name; adjust line thickness by changing the '1pt'
\moveleft\hoffset\vbox{BSc Software at Aalborg University}
\moveleft\hoffset\vbox{jens@staermose.com}
\moveleft\hoffset\vbox{+45 20 11 26 74}
\end{subfigure}
\fcolorbox{black}{white}{\includegraphics[width=0.1\resumewidth]{aal_087.jpg}}

\end{figure}

%----------------------------------------------------------------------------------------

\begin{resume}

%----------------------------------------------------------------------------------------
%	OBJECTIVE SECTION
%----------------------------------------------------------------------------------------
 
\section{OBJECTIVE}  

A newly educated bachelor student in Software, proficient in many languages and good academic knowledge within Computer Science.

%----------------------------------------------------------------------------------------
%	EDUCATION SECTION
%----------------------------------------------------------------------------------------

\section{EDUCATION}

{\sl HTX, Viby J.,} Mathematical \hfill 2009--2012 \\
{\sl Oure Højskole, Oure,} Skiinstructor \hfill 2012 \\
{\sl Bachelor of Science, Aalborg University (AAU),} Software \hfill 2013--2016
 
%----------------------------------------------------------------------------------------
%	COMPUTER SKILLS SECTION
%----------------------------------------------------------------------------------------
\section{LANGUAGES} 

{\sl Software:} 
C/C++, C\#, F\#, JAVA, HTML/CSS, JavaScript, Scala, LaTeX, Git, Postgres SQL \\
{\sl Natural languages:}
{Danish (Native), English (Fluent), German (Good command)}
 
%----------------------------------------------------------------------------------------
%	PROFESSIONAL EXPERIENCE SECTION
%----------------------------------------------------------------------------------------
\section{ACADEMICAL PROJECTS}

{\sl hotMap - Bachelor project} \hfill Spring 2016 \\
This project was a part of a larger project, aSTEP, involving around 40 people. aSTEP is a service for applications to share and store locations and mapdata that also includes mapmatching and user management.\\
I was a part of a smaller group of 5 people that developed an application that used the aSTEP core. We had contact with the security personnel on SmukFest, the second largest festival in Denmark, who were interested due to a series of problematic events at concerts from 2013 through 2015. In collaboration with them we developed an open source system that use wifi triangulation through smartphones wifi modules combined with specific event attendance information to detect crowd conditions and hazardous situations on festivals and large events to better aid security personnel detect, prevent and/or defuse dangerous and/or uncomfortable situations.

Project developed in Scala and JavaScript/CSS/HTML \hfill Graded: 12/A\\
hotMap: \url{http://daisy-git.cs.aau.dk/zanderdk/hotMap}\\
aSTEP: \url{http://daisy-git.cs.aau.dk/Yang/astep}

{\sl AHA - Automated Home Automation} \hfill Autumn 2015 \\
A system for imitating users’ behavioural patterns using hidden Markov Models and embedded devices We proposed a solution for imitating users’ behavioural patterns. Embedded devices were used to observe the state of the problem domain. Hidden Markov models were created, based on the history of states, and the Viterbi algorithm was used to predict a user’s behaviour. The solution is able to learn simple patterns and imitate appropriate actions within a reasonable time span.

Project developed in Java and Arduino C \hfill Graded: 10/B \\
\url{https://github.com/Hutli/AHA} \\

{\sl TLDR} \hfill Spring 2015 \\
A high-level, modular, functional first, domain specific programming language intended for natural and social scientists to model objects and interactions between objects in concurrent real-world systems with a full small-step semantics formal specification and formal syntax in BCNF form.

Project developed in F\# and \hfill Graded: 10/B \\
\url{https://github.com/Hutli/TLDR}

{\sl openPlaylist} \hfill Autumn 2014 \\
A triple-platform mobile application where users can vote for tracks at bars, private events or public spaces, in order to create a playlist fitting for the crowd. The project were developed in collaboration with the bar Fabrikken in central Aalborg, where it were tested and implemented. Usability tests with potential users were conducted and analysed.

Preject developed in C\# \hfill Graded: 12/A \\
\url{https://github.com/Hutli/openPlaylist}\\

\section{ACADEMICAL EXPERIENCE}

Due to the nature of the projects at AAU, I have 4 projects, each half a year long, worth of experience in Scrum and Agile development. I am also used to working in teams of 5-7 people, since I have been doing that for the last 3 years.

I have been a part of two projects with direct user contact. In these projects I learned how to conduct and design usability tests, interview techniques and general user contact.

The latest semester I was a part of a larger project involving around 40 people, split into 3-5 people groups. Therefore I have also tried to work in a larger organisational context with a lot of internal communication.

%----------------------------------------------------------------------------------------
%	EXTRA-CURRICULAR ACTIVITIES SECTION
%----------------------------------------------------------------------------------------

\section{EXTRA-CURRICULAR ACTIVITIES} 

AAUSAT6 {\it Aalborg University} \hfill 2016-- \\ 
A project conducted outside normal study time for creating the 6th satellite for the university. I have worked with
\begin{itemize}
\item Fault tolerance, deadlock avoidance and formal/model verification of the software mainly in UPPAAL
\item Formal/model verification of the communications protocol 
\item Version control and continuous build system for the whole project
\item An extra payload, in form of an extra processor to detect and count bit flips while being bit flip safe itself
\end{itemize}
I have together with Prof. René Rydhof Hansen created proposals for next years master theses, that, if choosen, should work together with SATLAB on any, or more, of the above mentioned areas.

%----------------------------------------------------------------------------------------
%	ADDITIONAL WORK
%---------------------------------------------------------------------------------------- 

\section{ADDITIONAL WORK}

Skiinstructor {\it Westendorf, Austria} \hfill 2012--2013 \\
After 6 months on Oure Højskole I was a certified skiinstructor (BSI1) and hereafter spent 4 months working full time in Westendorf. I have later further educated myself within the field and now have a full BSI (Basic Ski Instructor) at The Danish Ski School.

Gymnastics instructor {\it Odder, Denmark} \hfill 2010--2012 \\
I have trained two different local teams in gymnastics, two times a week over a period of two years.

%----------------------------------------------------------------------------------------

\end{resume}
\end{document}